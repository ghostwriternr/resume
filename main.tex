%%%%%%%%%%%%%%%%%%%%%%%%%%%%%%%%%%%%%%%%%
% Plasmati Graduate CV
% LaTeX Template
% Version 1.0 (24/3/13)
%
% This template has been downloaded from:
% http://www.LaTeXTemplates.com
%
% Original author:
% Alessandro Plasmati (alessandro.plasmati@gmail.com)
%
% License:
% CC BY-NC-SA 3.0 (http://creativecommons.org/licenses/by-nc-sa/3.0/)
%
% Important note:
% This template needs to be compiled with XeLaTeX.
% The main document font is called Fontin and can be downloaded for free
% from here: http://www.exljbris.com/fontin.html
%
%%%%%%%%%%%%%%%%%%%%%%%%%%%%%%%%%%%%%%%%%

%----------------------------------------------------------------------------------------
%	PACKAGES AND OTHER DOCUMENT CONFIGURATIONS
%----------------------------------------------------------------------------------------

\documentclass[a4paper,10pt]{extarticle} % Default font size and paper size

\usepackage{fontspec} % For loading fonts
\defaultfontfeatures{Mapping=tex-text}
% \setmainfont[SmallCapsFont = Fontin SmallCaps]{Fontin} % Main document font
% \fontspec{[FontAwesome.otf]}
\setmainfont[Path = ./Ubuntu/,  %% Optional; but UPDATE this if 
                         %% your font files are in a folder
 Extension = .ttf,
 UprightFont = *-Regular,
 BoldFont = *-Bold,
 ItalicFont = *-Italic,
 SmallCapsFont = *-Medium]
{Ubuntu}
\fontspec{[fontawesome-webfont.ttf]}

\usepackage{color}
\definecolor{primary}{RGB}{107, 16, 86}
\definecolor{secondary}{RGB}{0, 0, 0}

\usepackage{xunicode,xltxtra,url,parskip} % Formatting packages

\usepackage[usenames,dvipsnames]{xcolor} % Required for specifying custom colors

%\usepackage[big]{layaureo} % Margin formatting of the A4 page, an alternative to layaureo can be 
%\usepackage{fullpage}
\usepackage{geometry}
\geometry{a4paper,margin=0.45cm}
%\geometry{a4paper,left=20mm, top=20mm}
 %To reduce the height of the top margin uncomment: \addtolength{\voffset}{-1.3cm}

\usepackage{hyperref} % Required for adding links	and customizing them
%\definecolor{linkcolour}{rgb}{0,0.2,0.6} % Link color
\definecolor{linkcolour}{rgb}{0.3,0.3,0.3} % Link color
\hypersetup{colorlinks,breaklinks,urlcolor=linkcolour,linkcolor=linkcolour} % Set link colors throughout the document

\usepackage{titlesec} % Used to customize the \section command
\titleformat{\section}{\large\scshape\raggedright}{}{0em}{}[\titlerule] % Text formatting of sections
\titlespacing{\section}{0pt}{0pt}{0pt} % Spacing around sections

\usepackage{multicol}
\setlength{\columnsep}{0cm}

\usepackage{tabularx}

\usepackage{textcomp}

\usepackage{fontawesome}

\usepackage{enumitem}
\setlist[description]{%
  topsep=10pt,               % space before start / after end of list
  itemsep=1pt,               % space between items
%  font={\bfseries\sffamily\color{red}}, % if colour is needed
}

\def\arraystretch{1}
\renewcommand{\baselinestretch}{1.2}

\begin{document}

\pagestyle{empty} % Removes page numbering

%\font\fb=''[cmr10]'' % Change the font of the \LaTeX command under the skills section

%----------------------------------------------------------------------------------------
%	NAME AND CONTACT INFORMATION
%----------------------------------------------------------------------------------------
\begin{multicols}{3}
% \par{\centering\normalsize {\textsc{Undergraduate student at Indian Institute of Technology, Kharagpur}}\par}\normalsize
% \par{\centering\normalsize {\textsc{Department of Computer Science and Engineering}}\par}\normalsize
%\par{{\begin{center}Dual Degree, \emph{Computer Science and Engineering}\end{center}}}
\normalsize  \faGlobe\ {\href{https://ghostwriternr.me/}{\  ghostwriternr.me}}\\
\normalsize \faGithub\ {\href{https://github.com/ghostwriternr}{\  ghostwriternr}}\\
\normalsize  \faLinkedinSquare\ {\href{https://www.linkedin.com/in/naresh-ramesh}{\  naresh-ramesh}}\\
\columnbreak
\normalsize\par{\centering{\Huge\textsc{\textcolor{primary}{Naresh} Ramesh}}\par} % Your name
\par{\centering\normalsize{\textsc{Bangalore, India\\}}\hfill\par}
\columnbreak
\raggedright\hfill\normalsize \faEnvelope\ {\href{mailto:nareshmdu@gmail.com}{\  nareshmdu@gmail.com}}\\
\raggedright\hfill{\faPhone\ \  +91-7872833729}
\end{multicols}

%----------------------------------------------------------------------------------------
%	WORK EXPERIENCE 
%----------------------------------------------------------------------------------------

\vspace{-0.7cm}
\section{\textcolor{primary}{Work Experience}}

\begin{tabularx}{\linewidth}{ l | X }

\textsc{Current} & \textbf{Software Engineer}\hfill\textbf{Intuit, India}\\
\textsc{Aug 18} & {- Migrated new signups for QBSE US to QBO's new billing platform with support for async APIs, cross-channel (Web, iOS and Android) billing capabilities \& Redis offer caching for mobile offers.}\\
& {- Took ownership of the new billing platform's most used API and added clean abstraction, error and performance monitoring along with some enhancements for new business requirements.}\\
& {- Identified \& filled monitoring gaps in the billing platform's highly critical signup API with an extensible framework written in Java to publish metrics to Wavefront and analyze real-time data with region \& offering-wise alerts.}\\
& {- Reduced support turnaround time by validating frequent customer care call drivers with a Spring + React tool.}\\
& {- Worked as India Site Leader for Open Source, evangelising and building the new Open Source program at Intuit.}\\
\multicolumn{2}{c}{} \\

\textsc{Jul 18} & \textbf{Google Summer of Code Intern}\hfill\textbf{gRPC (Cloud Native Computing Foundation)}\\
\textsc{May 18}& {- Added Bazel support for building the gRPC Python codebase, with custom Skylark rules for the Cython modules.}\\
& {- Extended support to Google's internal CI (Kokoro) as well, allowing incremental build \& test runs with Bazel.}\\
% & {- Upgraded the Google-internal CI (Kokoro) for gRPC Python to use the Bazel builds rather than handwritten Python jobs.}\\
& {- Detected \& reported issues in rules\_python's experimental pip support and module level logging specification for python frameworks.}\\
\multicolumn{2}{c}{} \\

\textsc{Nov 17} & \textbf{Remote Software Engineer Intern}\hfill\textbf{Weoptit, Finland}\\
\textsc{Aug 17}& {- Built a pipeline to parse and store data from Finnish e-commerce sites to be used for AI based pricing optimisation.}\\
& {- Wrote the parsers for 21 websites using Python, each deployed to production to periodically fetch data.}\\
\multicolumn{2}{c}{} \\

\textsc{Jul 17} & \textbf{Software Engineer Intern}\hfill\textbf{Intuit, India}\\
\textsc{May 17}& {- Wrote an end-to-end GraphQL API service in Java for opting out of QuickBooks Online's subscription.}\\
& {- Deployed a Machine Learning model to help Intuit predict potential paid customers for QBO from trial data.}\\
& {- Reduced false positive error reports by 30\% from other teams to our team with a revamped internal tool.}\\
\multicolumn{2}{c}{} \\

\textsc{Jun 16} & \textbf{Software Engineer Intern}\hfill\textbf{ezDI, India}\\
%\textsc{May 2016}& {- Worked on integrating a Business Intelligence tool that aggregates data from all of ezDI's products for easy analytics.}\\
\textsc{May 16}& {- Automated migration \& replication of real-time data from AWS RDS to Redshift, optimized for full-table queries.}\\
& {- Implemented proof-of-concepts to embed a Business Intelligence (BI) solution in the enterprise analytics platform.}\\
\multicolumn{2}{c}{} \\

\textsc{Apr 16} & \textbf{Software Engineer} \textsc{- Image Processing}\hfill\textbf{Autonomous Underwater Vehicle Research Group}\\
% \textsc{Feb 2015} & \textbf{Guide: }\textmd{\href{http://iitkgp.ac.in/fac-profiles/showprofile.php?empcode=aWmdU}{Professor C. S. Kumar}}\\
% \textsc{Feb 15} & {- Worked on the image processing team at IITKGP's autonomous underwater vehicle research group.}\\
\textsc{Feb 15} & {- Added image processing logic for Buoy detection and path following underwater using OpenCV on ROS.}\\
& {- Wrote Neural Network based adaptive image segmentation for the bot to adapt to dynamic lighting conditions.}\\
& {- Created an AngularJS frontend for remotely monitoring and controlling the bot's image parameters.}\\
\end{tabularx}

%----------------------------------------------------------------------------------------
%	EDUCATION
%----------------------------------------------------------------------------------------

\vspace{0.1cm}
\section{\textcolor{primary}{Education}}

\begin{tabular}{r|p{17.5cm}}	
\textsc{2013-2018} & Bachelors + Masters in \textbf{Computer Science and Engineering} - Indian Institute of Technology, Kharagpur\\
%\hfill\textsc{Cgpa}: 7.59/10.0\\
% &\textbf{Indian Institute of Technology}, Kharagpur\\
&Teaching assistant for Compilers (CS31003) and Data Structures (CS19001)\\
%&\textbf{Coursework: }{Programming and Data Structures, Algorithms-I \& II, Software Engineering, Compilers, Switching Circuits, Operating Systems, Computer Networks, Information Retrieval, Database Management Systems, Theory of Computation, Machine Learning, Image Processing, Advanced Graph Theory}
\end{tabular}

%----------------------------------------------------------------------------------------
%	SKILLS 
%----------------------------------------------------------------------------------------

\vspace{0.1cm}
\section{\textcolor{primary}{Technical Skills}}

\begin{tabular}{r|p{15cm}}
\textsc{Languages} & Java, C++, Python, Javascript, HTML/CSS\\
% & \textit{Competent:} C, Golang \\
\textsc{Libraries / Frameworks} & Spring, Spring Boot, Hibernate, React, Redux, NodeJS, Maven, Bazel, Junit, Mockito, Chai, Intern, Scikit-learn, OpenCV, Flask\\
\textsc{Databases} & MySQL, MongoDB, Redis\\
\textsc{Systems / Platforms} & Apache Kafka, Docker, Git, Linux, Android\\
%\textsc{Markup / Templating} & HTML, CSS, Sass
\end{tabular}


%----------------------------------------------------------------------------------------
%	Projects
%----------------------------------------------------------------------------------------

% \section{\textcolor{primary}{Projects}}
% \vspace{-0.6cm}
% \begin{tabular}{p{19.7cm}}
% % \fontsize{9}{12}\selectfont{
% \begin{description}[style=nextline, font=$\bullet$\hspace{2mm}\normalsize]
% %  \item[Learning to extract comparison points of entity pairs from Wikipedia articles] Worked on a novel comparative text mining task using relational tuples to model and measure semantic commonality for two given documents and tabulating them. This work was published as a poster at JCDL 2018.
%  \item[Parsing and extracting metadata from medical prescription images] Built the React frontend and Python Flask API for a medical prescription parsing software, with Tesseract for OCR.
% %  \item[Compiler for Tiny C (A tiny subset of the C language)] Built a compiler for Tiny C, a self-defined subset of the C language, using Compiler principles and techniques. The compiler itself was written in C++ with Flex and Bison for Lexical Analysis and Semantic parsing.
% %  \item[Mini Linux Shell] Implemented some features of the Linux Shell including redirection, piping and terminal mirroring using the C language.
% %  \item[Lyrics generator using Neural Networks] Wrote a lyrics generator using TensorFlow that generates a new song in an artist's style. A Long Short Term Memory (LSTM) Neural Network learns the artists' styles of writing including words, rhymes, chorus patterns, etc. for the task.
% %  \item[Data extraction from biomedical literature for automating systematic reviews] Worked on feature detection of a particular class of text (specifically, inclusion and exclusion criteria for patients) from a huge collection of biomedical literature using NLP Techniques with high precision and recall.
% %  \item[Selene] Built a social music-recommendation Android app that analyzes songs liked by a users' friends on YouTube and recommends them to the user as a trending feed with added song metadata using the MusixMatch API.
% %  \item[Retrieving salient sentences from Reddit AMAs] Built a summariser for /r/IAmA to create reader-friendly digests that are clustered by topic and presented to the user.
% %  \item[Graph Extractor] Built a graph extractor that uses image processing techniques to detect multivariate graphs in any given PDF and tabulates them autonomously, taking into consideration features like axis values, scales and legends.
%  \item[MetaKGP dashboard] Built and deployed an Open Source NUS-Mods style dashboard for my university with several utilities for the student community like smart timetable, news aggregator, question paper portal, professor finder to name a few.
% % \item[Studious] Built a complete course management system that supported authentication \& authorization, User Access Control for 4 different types of users, real-time messaging with notifications (using socket.io), calendar support, etc.
% %  \item[Medical Lab Automation System] Developed software using JAVA Swing for a Medical Lab Automation System which handles and automates all requests of the management and patients.
% \end{description}
% \end{tabular}
% % \vspace{-0.3cm}
% % \section{\textcolor{primary}{Activities \& Leadership}}

% % \begin{itemize}[leftmargin=0.55cm, rightmargin=0.2cm, label={\Large\textbullet}]
% % \item Organisation Maintainer at MetaKGP to moderate, mentor and work with the student community on several open-source projects and a self-hosted Wiki for IIT Kharagpur.
% % \item Won 3rd place and 5th place at tech competitions Inter-IIT and OpenSoft (LBS) respectively as team captain of 10 students.
% % \item Organised hackathons \& coding challenges as a part of BitWise and managed a team of 15 students to raise sponsorship.
% % \end{itemize}

% % \begin{tabularx}{\linewidth}{ l | X }

% % \textsc{Current} & \textbf{Organisation maintainer}\hfill\textbf{MetaKGP}\\
% % & {- Moderate several technical projects and the MetaKGP Wiki built by the student community for IIT KGP.} \\
% % & {- Currently working with student developers on a NUS-Mods style dashboard for IIT KGP with several utilities for the community like smart timetable, news aggregator, question paper portal and professor finder to name a few.}\\
% % \multicolumn{2}{c}{} \\

% % \textsc{Dec 16} & \textbf{StoI} \textsc{(SMS to Internet)}\hfill\textbf{Pragyan Hackathon `17}\\
% % & {- Made an android application for basic internet access like Google Maps navigation, Duckduckgo quick search, Zomato reviews, etc. without a data connection. Communication with the server was done using Twilio's SMS APIs.}\\
% % \multicolumn{2}{c}{} \\

% % \textsc{Apr 16} & \textbf{Data Extractor for 2D plots}\hfill\textbf{OpenSoft '16}\\
% % & {- Built a graph extractor that detects multi-variable graphs in any given PDF and tabulates them autonomously taking into consideration features like axis values, scales and legends.}\\
% % & {- Was primarily responsible for detecting and scaling the ticks from the colour segregated image and scale it appropriately using the axis values and generate the plot. Was also solely responsible for generating the table structure using Python libraries and to fully build a working GUI for the application on Java Swing.}\\
% %\multicolumn{2}{c}{} \\

% %\textsc{Mar 15} & \textbf{Campus Connexions}\hfill\textbf{Microsoft Code.Fun.Do 2015}\\
% %& {- Developed an intra-college social networking app with real time feed from registered users that would serve as a platform for official and unofficial announcements related to the college.}\\
% % Windows Azure was used for database management and development was primarily done with Visual Studio.}\\
% % & {- Was solely responsible for building the complete front end for the app, and to connect with Azure to dynamically load content in the news feed.}\\
% %\multicolumn{2}{c}{} \\

% %\textsc{Dec 2014} & \textbf{Object Follower Robot}\hfill\textbf{Technology Robotix Society, IIT Kharagpur}\\
% %& {- Implemented image detection algorithms using openCV for a WSAD robot which can follow a specified path using the directives sent by overhead camera whose recorded images were processed and movement instructions generated.}
% % \end{tabularx}

%----------------------------------------------------------------------------------------
%	Publications
%----------------------------------------------------------------------------------------

% \vspace{-0.6cm}
\section{\textcolor{primary}{Publications}}
\vspace{-0.6cm}
\begin{tabular}{p{19.7cm}}
% \fontsize{9}{12}\selectfont{
\begin{description}[style=nextline, font=$\bullet$\hspace{2mm}\normalsize]
 \item[Learning to extract comparison points of entity pairs from Wikipedia articles\hfill{JCDL 2018}] Worked on a novel comparative text mining task using relational tuples to model and measure semantic commonality for two given documents and tabulating them. Published at the Proceedings of the 18th ACM/IEEE on Joint Conference on Digital Libraries (JCDL), 2018.
\end{description}
\end{tabular}

%----------------------------------------------------------------------------------------
%	POSITIONS OF RESPONSIBILITY
%----------------------------------------------------------------------------------------

%\vspace{-0.3cm}
%\section{Positions of Responsibility}

%\begin{tabular}{r|p{17.5cm}}
%\textsc{Current} & \textbf{Maintainer}, MetaKGP\\
%\textsc{Current} & \textbf{Chief Editor}, Technology Literary Society, IIT Kharagpur \\
%& {- Managing the content and design team of the society.}\\
%& {- Writer in the English Team, and working as a senior editor for all English publications.}\\
%\textsc{Apr 2017} & \textbf{Captain}, Team LBS, OpenSoft 2017\\
%\textsc{Apr 2016} & \textbf{General Secretary}, CodeClub, IIT Kharagpur \\
%\textsc{Apr 2015} & \textbf{Secretary}, CodeClub, IIT Kharagpur \\
%& {- Part of the managing team, leading a group of 25 students.}\\
%& {- Conducted several events, including Microsoft code.fun.do and BITWISE, the Annual Departmental Fest of the Department of Computer Science and Engineering, alongside several fortnightly competitive coding competitions within the campus.}\\
%& {- Designed the software for updating the leaderboards in real-time during competitions as a part of BITWISE '15.}\\
%\textsc{Apr 2015} & \textbf{Core Team Member}, Google Students Club, IIT Kharagpur
%& {- Organized multiple workshops and events, primarily focused on Android Development, in association with Google.}\\
%& {- Conducted a workshop on the {\href{https://www.polymer-project.org/0.5/}{Polymer Project}}, which received high levels of}\\ & {participation.}\\
%\end{tabular}

%----------------------------------------------------------------------------------------
%	COURSEWORK
%----------------------------------------------------------------------------------------

% \section{Coursework
% \hfill\small\textsc{(T)heory and (L)aboratory}}

% \begin{multicols}{2}
% - Programming and Data Structures (T/L) \\
% - Discrete Structures \\
% - Algorithms - II \\
% - Switching Circuits (T/L) \\
% - Operating Systems (T/L) \\
% - Computer Networks (T/L) \\
% - Machine Learning * \\
% - Image Processing * \\
% - Algorithms-I (T/L) \\
% - Software Engineering (T/L) \\
% - Formal Languages and Automata Theory \\
% - Compilers (T/L) \\
% - Database Management Systems (T/L) \\
% - Information Retrieval \\
% - Object Oriented Software Design * \\
% - Advanced Graph Theory *
% \end{multicols}
% {\itshape{Currently Studying:}}\\

%----------------------------------------------------------------------------------------
%	ACHIEVEMENTS
%----------------------------------------------------------------------------------------

% \section{Scholastic Achievements}

% %\begin{multicols}{2}
% - Secured 98.11 percentile in JEE Advanced 2013 \\
% - Secured 99.33 percentile in JEE Mains 2013 \\ 
% - Secured AIR 415 in ACM ICPC – Amritapuri online round, 2014
% %\end{multicols}

%----------------------------------------------------------------------------------------

%\newpage
%----------------------------------------------------------------------------------------

\end{document}
